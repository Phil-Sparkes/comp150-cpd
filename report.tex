% Please do not change the document class
\documentclass{scrartcl}

% Please do not change these packages
\usepackage[hidelinks]{hyperref}
\usepackage[none]{hyphenat}
\usepackage{setspace}
\doublespace

% You may add additional packages here
\usepackage{amsmath}

% Please include a clear, concise, and descriptive title
\title{CPD Report}

% Please do not change the subtitle
\subtitle{COMP150 - CPD Report}

% Please put your student number in the author field
\author{1608305}

\begin{document}

\maketitle

\section{Introduction}

To work with computers preferably in something that could hold my interest like making games.
Learning how to program, plan to make a portfolio of all coding and computing related projects I have worked on, learning about the games industry and the agile ways. I will be volunteering for computing related events to acquire different skills and contacts in the industry.
Some of the challenges I have come across are: explaining code to others, time management, inefficient coding practices, presenting pitches and problems with Github.
To help with explaining code to others I would like to teach someone how to make a simple game using python. For the time management I plan on making a schedule that I will follow when we get back after Christmas break. To help make my code more efficient I will read the book on “Clean Code” over Christmas. For improving my presentation skills I plan on videoing myself doing a presentation and then self-evaluating my performance to see how I could improve it. Finally for the Github I plan on learning how to implement Travis and branching systems correctly.



\section{Explaining code to others}

For programming practices it is a key skill to be able to explain how your code works to others in a way that they can understand. Being able to explain how your program works to fellow programmers will mean that they can maintain the code as well. This can be done by use of well-constructed comments or face to face communication. I feel like this skill with help me especially because I have been in situations where I have had to explain my code and struggled to do it in a way that the other person could understand. Without my colleges knowing how the code works they cannot add to the code or maintain the code. This effected some of my work when working in teams as I was the only one able to do the programming. To overcome this I have decided to improve my teaching skills, being able to teach someone how to code will make it so they can get an understanding of what my code is doing. I plan to do this over the Christmas break offering to teach someone how to make a basic game on python in a week course for free. After I would like to show them some code that I have done and see if I can explain it in a way they would understand and ask for feedback on the comments.

\section{Time management - Keeping too the nine hours a week}

Being able to manage your time with professional game development is a must. Keeping to schedule and making sure all the features are implemented on time. Some of my work had to completed the night before it was due in thus making it rushed and not to the quality I wanted it to be. An action I am taking to try and prevent this from happening again will be to schedule when I work on which module while doing at least nine hours every week. To make this time bound I have decided to test it during the first two weeks back so I can review the schedule and see what effect it has on my workload. I plan to keep this going after the two weeks so that I am not rushing my work last minute.


\section{Efficient coding practices}

When developing games following efficient coding practices is a must as it leads to easier maintainability and faster computing times. I feel like most of the code I have done could be improved in some fashion to be more efficient. I want to learn how to code in the most efficient way to better my programming skills leading to better programs. I feel like practice is the best way to do this but also learning from others. I have decided to read the book “clean code” over the Christmas break to gain a better understanding of good programming practice. I plan to implement what I have learnt from this book into future programs that I will write.

\section{Presenting pitches}

Another key skill in the games industry is the ability to pitch game ideas to investors. Being able to successfully pitch game concepts to developers will mean that you are more likely to receive investment. During the course so far we have had to do a number of pitches in which I am not completely satisfied with my performance. I feel like this is one of the skills that I need to improve. I plan to video myself pitching a game idea and then watch it back and review the quality of the pitch, what I have done well and what I have done not so well. I will take this information and then incorporate it into the next pitch. I will do one video every day for a week hopefully improving my pitching skills.


\section{Using Github (Travis and branching)}

Github or similar version control systems are widely used in game and software development. Knowing how to get the most out of it would be a useful skill to have. During my groups game development project we tried to incorporate “Travis” into our Github repository so it could run test scripts to check for bugs. We never managed to get it working correctly so ended up not using it. I am interested to see what it can be used for so I want to implement it successfully. Another feature of Github is branching, during the game project I mostly worked on the master branch which is bad practice. Learning how to make effective use of the branching system would be a key skill. To learn these skills I will be looking at documentation and tutorials on how to get these features working during the first two weeks back. I will consider it a success when I manage to get both Travis and the branching working correctly.


\section{Conclusion}

I have planned to do these actions over Christmas and the first week’s back to improve my future work. Improving each of these key skills will help bring me closer to my intended career goal. I will mainly work on these key skills over Christmas and on the first weeks back so that I am ready for the next section of the course and be able to implement these skills to help improve my projects.

\bibliographystyle{ieeetran}
\bibliography{references}

\end{document}
